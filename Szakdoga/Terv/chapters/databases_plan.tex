\chapter{Adatbázis táblák tervei} % User guide
\label{ch:databases_plan}

\section{Felhasználók autentikációja}

\begin{table}[htb]
	\centering
	\begin{tabular}{ |c|c|c| }
		\hline
		\multicolumn{3}{|c|}{Identity táblázat}\\
		\hline
		Név & Típus & Cél \\
		\hline
		Password  & String  & Jelszó titkosított tárolása.  \\
		\hline
		UserName & String & user nevek tárolása azonosítás céljából\\
		\hline
	\end{tabular}
	\caption[Rövid cím a táblázatjegyzékbe]{Vivamus ac arcu fringilla, fermentum neque sed, interdum erat. Mauris bibendum mauris vitae enim mollis, et eleifend turpis aliquet.}
	\label{tab:example-2}
\end{table}


\begin{table}[htb]
	\centering
	\begin{tabular}{ |c|c|c| }
		\hline
		\multicolumn{3}{|c|}{User táblázat}\\
		\hline
		Név & Típus & Cél \\
		\hline
		UserName  & String  & bejelentkezéshez, egyedi  \\
		\hline
		NickName & String & játékon belól használt név, egyedi\\
		\hline
		Messages & <(String, Date)> & játékon belüli értesítések\\
		\hline
		FriendList & <Friend> & barát lista\\
		\hline
	\end{tabular}
	\caption[Rövid cím a táblázatjegyzékbe]{Vivamus ac arcu fringilla, fermentum neque sed, interdum erat. Mauris bibendum mauris vitae enim mollis, et eleifend turpis aliquet.}
	\label{tab:example-2}
\end{table}

\begin{table}[htb]
	\centering
	\begin{tabular}{ |c|c|c| }
		\hline
		\multicolumn{3}{|c|}{Friend táblázat}\\
		\hline
		Név & Típus & Cél \\
		\hline
		RequestFrom  & User  & Az a felhasználó aki a meghívást küldte.  \\
		\hline
		RequestTo & User & Az a felhasználó, aki kapta a barátmeghívást.\\
		\hline
		IsAccepted & Bool & Igaz ha a meghívott fél elfogadta a meghívást.\\
		\hline
	\end{tabular}
	\caption[Rövid cím a táblázatjegyzékbe]{Vivamus ac arcu fringilla, fermentum neque sed, interdum erat. Mauris bibendum mauris vitae enim mollis, et eleifend turpis aliquet.}
	\label{tab:example-2}
\end{table}

\section{Kalandjáték}

\begin{table}[htb]
	\centering
	\begin{tabular}{ |c|c|c| }
		\hline
		\multicolumn{3}{|c|}{Game táblázat}\\
		\hline
		Név & Típus & Cél \\
		\hline
		Map  & <Field>  & a játékhoz tartozó térkép  \\
		\hline
		TargetField & Field & célmező, ahova a játékosnak el kell jutnia a nyeréshez\\
		\hline
		PlayCounter & Int & hányszor játszottak vele\\
		\hline
		Visibility & Visibility & kik láthatják \newline (csak én/barátok/bejelentkezettek/midnenki)\\
		\hline
	\end{tabular}
	\caption[Rövid cím a táblázatjegyzékbe]{Vivamus ac arcu fringilla, fermentum neque sed, interdum erat. Mauris bibendum mauris vitae enim mollis, et eleifend turpis aliquet.}
	\label{tab:example-2}
\end{table}

\begin{table}[htb]
	\centering
	\begin{tabular}{ |c|c|c| }
		\hline
		\multicolumn{3}{|c|}{Field táblázat}\\
		\hline
		Név & Típus & Cél \\
		\hline
		Coordinate  & (Int, Int)  & Hol helyezkedik el a térképen  \\
		\hline
		Text & String & A térképkockához tartozó történet.\\
		\hline
		MultipleChoice & MultipleChoice & térképkocka akciója\\
		\hline
		LuckTrial & LuckTrial & térképkocka akciója\\
		\hline
		Illustration & Picture& térképkockához tartozó illusztráció\\
		\hline
	\end{tabular}
	\caption[Rövid cím a táblázatjegyzékbe]{Vivamus ac arcu fringilla, fermentum neque sed, interdum erat. Mauris bibendum mauris vitae enim mollis, et eleifend turpis aliquet.}
	\label{tab:example-2}
\end{table}

Mező események
\begin{itemize}
	\item Feleletválasztás $\rightarrow$ semmi / vége a játéknak / teleportál
	\item Szerencsepróba $\rightarrow$ semmi / vége a játéknak / teleportál
	\item Teleportálás
	\item Semmi
\end{itemize}