\chapter{Adatbázis táblák tervei} % User guide
\label{ch:databases_plan}

Ebben a fejezetben szerepelnek egy kalandjáték adatainak letárolásához szükséges adatbázistáblák és típusok tervei.

\section{Felhasználók autentikációja}

\begin{table}[htb]
	\centering
	\begin{tabular}{ |c|c|c| }
		\hline
		\multicolumn{3}{|c|}{Identity táblázat}\\
		\hline
		Név & Típus & Cél \\
		\hline
		Password  & String  & Jelszó titkosított tárolása.  \\
		\hline
		UserName & String & user nevek tárolása azonosítás céljából\\
		\hline
	\end{tabular}
	\caption[Identity adatbázistábla]{Az Identity c. adatbázistábla felépítése.}
	\label{tab:identity}
\end{table}


\begin{table}[htb]
	\centering
	\begin{tabular}{ |c|c|c| }
		\hline
		\multicolumn{3}{|c|}{User táblázat}\\
		\hline
		Név & Típus & Cél \\
		\hline
		NickName & String & játékon belól használt név, egyedi\\
		\hline
		Messages & <(String, Date)> & játékon belüli értesítések\\
		\hline
		FriendList & <Friend> & barát lista\\
		\hline
	\end{tabular}
	\caption[User adatbázistábla]{A User c. adatbázistábla felépítése.}
	\label{tab:user}
\end{table}

\begin{table}[H]
	\centering
	\begin{tabular}{ |c|c|c| }
		\hline
		\multicolumn{3}{|c|}{Friend táblázat}\\
		\hline
		Név & Típus & Cél \\
		\hline
		RequestFrom  & User  & Az a felhasználó aki a meghívást küldte.  \\
		\hline
		RequestTo & User & Az a felhasználó, aki kapta a barátmeghívást.\\
		\hline
		IsAccepted & Bool & Igaz ha a meghívott fél elfogadta a meghívást.\\
		\hline
	\end{tabular}
	\caption[Friend adatbázistábla]{A Friend c. adatbázistábla felépítése.}
	\label{tab:friend}
\end{table}

\section{Kalandjáték}

\subsection{Egy játékkal kapcsolatos adatok tárolása}

\begin{table}[H]
	\centering
	\begin{tabular}{ |c|c|c| }
		\hline
		\multicolumn{3}{|c|}{Game táblázat}\\
		\hline
		Név & Típus & Cél \\
		\hline
		Map  & <Field>  & a játékhoz tartozó térkép  \\
		\hline
		TargetField & Field & célmező, ahova a játékosnak el kell jutnia a nyeréshez\\
		\hline
		PlayCounter & Int & hányszor játszottak vele\\
		\hline
		Visibility & Visibility & kik láthatják \newline (csak én/barátok/bejelentkezettek/midnenki)\\
		\hline
		Title & String & A játék címe. \\
		\hline
		TableSize & Int & A játék térképének mérete (pl. 3x3, 4x4, stb.).\\
		\hline
	\end{tabular}
	\caption[Game adatbázistábla felépítése]{Egy játék letárolásának módja.}
	\label{tab:game}
\end{table}

Visibility felsorolási típus
\begin{compactitem}
	\item Owner, Friends, LoggedIn, Everyone
\end{compactitem}

\subsection{Egy mező adatainak letárolása}

\begin{table}[H]
	\centering
	\begin{tabular}{ |c|c|c| }
		\hline
		\multicolumn{3}{|c|}{Field táblázat}\\
		\hline
		Név & Típus & Cél \\
		\hline
		Coordinate  & FieldCoordinate  & Hol helyezkedik el a térképen  \\
		\hline
		Text & String & A térképkockához tartozó történet.\\
		\hline
		FieldEvent & FieldEvent & térképkocka akciója\\
		\hline
		Illustration & Picture & térképkockához tartozó illusztráció hivatkozása\\
		\hline
		MapPieceType & MapPieceType & Milyen irányokba lehet továbbhaladni a mezőről.\\
		\hline
	\end{tabular}
	\caption[Field adatbázistábla]{A Field c. adatbázistábla felépítése.}
	\label{tab:field}
\end{table}

\begin{table}[H]
	\centering
	\begin{tabular}{ |c|c|c| }
		\hline
		\multicolumn{3}{|c|}{A MapPieceType típus}\\
		\hline
		Név & Típus & Cél \\
		\hline
		IsRightWay  & Boolean  & Igaz, ha \textbf{jobbra} lehet haladni az adott mezőről.  \\
		\hline
		IsLeftWay & Boolean & Igaz, ha \textbf{balra} lehet menni az adott mezőről.\\
		\hline
		IsUpWay & Boolean & Igaz, ha \textbf{felfelé} lehet haladni az adott mezőről.\\
		\hline
		IsDownWay & Boolean & Igaz, ha \textbf{lefelé} lehet továbbhaladni a mezőről.\\
		\hline
	\end{tabular}
	\caption[FielCoordinate típus]{A FieldCoordinate c. típus felépítése.}
	\label{tab:field}
\end{table}

\begin{table}[H]
	\centering
	\begin{tabular}{ |c|c|c| }
		\hline
		\multicolumn{3}{|c|}{A FieldCoordinate típus}\\
		\hline
		Név & Típus & Cél \\
		\hline
		HorizontalCoordinate  & Int  & Vízszintes koordináta.  \\
		\hline
		VerticalCoordinate & Int &Függőleges koordináta.\\
		\hline
	\end{tabular}
	\caption[FielCoordinate típus]{A FieldCoordinate c. típus felépítése.}
	\label{tab:field}
\end{table}

\subsection{A képek tárolása}

\begin{itemize}
	\item Lesz egy külön picture táblázat, amelyből az ID alapján be lehet majd tölteni az illusztrációkat. (Kezdetben a default illusztrációk közül lehet majd választani, később esetleg sajátot is fel lehet majd tölteni ha megfelelő a formátuma.)
	
	\item Külön táblázatban lesznek tárolva a térképkockák képei amelyek a tervezés során jelennek meg.
\end{itemize}

\subsection{Mezőesemények}

\begin{table}[H]
	\centering
	\begin{tabular}{ |c|c|c| }
		\hline
		\multicolumn{3}{|c|}{FieldEvent típus}\\
		\hline
		Név & Típus & Cél \\
		\hline
		IsTrial  & Boolean  & Igaz, ha valamilyen próba van a mezőn.\\
		\hline
		TrialID & Int & Ha valamilyen próba van a mezőn, annak ID-ja.\\
		\hline
	\end{tabular}
	\caption[FieldEvent típus]{A FieldEvent c. típus felépítése.}
	\label{tab:field}
\end{table}


\begin{table}[htb]
	\centering
	\begin{tabular}{ |c|c|c| }
		\hline
		\multicolumn{3}{|c|}{Trial táblázat}\\
		\hline
		Név & Típus & Cél \\
		\hline
		ID & Int & Azonosító, amin keresztül hivatkozhatunk rá. \\
		\hline
		Alternatives & <Alternative> & Lehetséges bekövetkező események.\\
		\hline
		TrialType & TrialType & a próba típusa (szerencse / játékos választ)\\
		\hline
	\end{tabular}
	\caption[Trial adatbázistába]{A Trial adatbázistábla felépítése.(Az EventId egy Trial táblázatbeli elemre hivatkozik.)}
	\label{tab:trial}
\end{table}

A \textbf{TryalType felsorolási típus}
\begin{compactitem}
	\item LuckTrial, MultipleChoice
	\item Megjegyzés: gyakorlatilag azt adja meg, hogy mi alapján választunk a bekövetkezhető alterntívák közül. Szerencsepróba esetén minden alternatíva azonos eséllyel következhet be, ezt kisorsoljuk. A választás esetén a játékos választhatja ki egy listából az opciók szövege alapján, hogy mi történjen (persze ekkor még nem tudja annak következményét).
\end{compactitem}


\begin{table}[H]
	\centering
	\begin{tabular}{ |c|c|c| }
		\hline
		\multicolumn{3}{|c|}{Az Alternative típus}\\
		\hline
		Név & Típus & Cél \\
		\hline
		Text  & String  & Az opció szövege.  \\
		\hline
		TrialResult & TrialResult & Az opció kiválasztása esetén történő esemény.\\
		\hline
	\end{tabular}
	\caption[Alternative típus]{Az Alternative típus felépítése.}
	\label{tab:alternative}
\end{table}

Megjegyzés: későbbiekben esetleg út lezárása, de ez esetben biztosítani kell, hogy ne záruljon le minden út, vagy ha mégis, akkor az a játék végét jelentse.

\begin{table}[H]
	\centering
	\begin{tabular}{ |c|c|c| }
		\hline
		\multicolumn{3}{|c|}{A TrialResult típus}\\
		\hline
		Név & Típus & Cél \\
		\hline
		IsTeleport  & Boolean  & Ha igaz, tepeortálunk, ha nem, nincs semmi.  \\
		\hline
		FieldCoordinate & FieldCoordinate & Ha teleportálunk, akkor hová.\\
		\hline
		Text & String & A történés leírása. \\
		\hline
	\end{tabular}
	\caption[Alternative típus]{Az Alternative típus felépítése.}
	\label{tab:alternative}
\end{table}

